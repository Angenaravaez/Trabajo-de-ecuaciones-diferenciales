\documentclass[12pt,letterpaper,oneside]{article}
\usepackage[T1]{fontenc}
\usepackage[utf8]{inputenc}
\usepackage{enumerate}
\usepackage[spanish]{babel}
\usepackage{amsmath}
\usepackage{amsfonts}
\usepackage{amssymb}
\usepackage[left=24.5mm,top=24.5mm,bottom=24.5mm,right=24.5mm]{geometry}
\usepackage{fancyhdr}
\pagestyle{fancy}
\fancyhead{}
\fancyhead[l]{SOLUCIÓN DE ED LINEALES POR SERIES DE POTENCIAS}
\usepackage{graphicx}
\graphicspath{{IMAGENES/}}
\begin{document}
\begin{center}
\begin{figure}
\begin{center}
\includegraphics[scale=0.4]{escudo}
\end{center}
\end{figure}
UNIVERSIDAD DEL CAUCA \\
Facultad de ingeniería civil\vspace{3.0cm}\\

CURSO DE VERANO ECUACIONES DIFERENCIALES\\
Programa: ingeniería civil\vspace{1.5cm}\\

"TRABAJO SOBRE SOLUCIÓN DE ECUACIONES DIFERENCIALES LINEALES MEDIANTE SERIES\\
(Solución mediante series de potencia en un punto ordinario)"\vspace{3cm}\\

Autora:\\
Angélica María Narváez Calvache\vspace{1.5cm}\\

Docente:\\
Jhonatan Collazos Ramirez\vspace{3.0cm}\\ 

Agosto de 2022\\

\end{center}
\thispagestyle{empty}
\newpage
\setcounter{page}{1}
\section{Introducción}
Este artículo inicia con un repaso de series de potencia en el cual se brindan los conceptos necesarios para la comprensión de tema, además, como contenido principal se realiza la definición de la solución de ecuaciones diferenciales lineales mediante series de potencia en un punto ordinario, para ello se aplican diferentes referencias bibliograficas. Posteriormente la exposición de un ejercicio de aplicaión y una progamación haciendo uso del software Matlab.
\section{Objetivos}
\begin{itemize}
\subsection{Objetivo principal}
 \item Aplicar el método de solución de ecuaciones diferenciales lineales mediante series de potencia en el caso especifico de un punto ordinario.
\subsection{Objetivos especificos}

 \item Identificar una serie de potencia.\\
\item Aplicar conceptos para identificar ED lineales que se puedan solucionar mediante el metodo de series de potencia.
\end{itemize}
\section{Metodología}
 Para efectuar este articulo, se realizó un tipo de recopilación y analisis de distintas referencias, además, la aplicación de los conceptos en un ejercicio en el software de programación Matlab.
 
\section{Resumen}

 Las series de potencia ayudan a resolver y relacionar con diversos problemas tanto de la matemática pura como de la matemática aplicada. Para la solución de ecuaciones diferenciales y por lo tanto a la creación de modelos matemáticos. Zill(2006)\vspace{1.0cm}\\
El método de serie de potencias para resolver una ED lineal con coeficientes variables con frecuencia se describe como “método de coeficientes indeterminados de series”. En resumen, la idea es la siguiente: sustituimos $y = \Sigma{^\infty_{n=0}}C_nx^n$ en la ecuación diferencial, se combina la serie como se muestra en el ejemplo de suma de series de potencias  y luego se igualan los coeficientes del miembro derecho de la ecuación para determinar los coeficientes $c_n$. Pero como el miembro derecho es cero, el último paso requiere, por la propiedad de identidad en la lista de propiedades de repaso, de serie de potencias, que todos los coeficientes de $x$ se deban igualar a cero. Esto no
significa que los coeficientes son cero; esto no tendría sentido después de todo; el teorema 1 garantiza que se pueden encontrar dos soluciones.

\section{Repaso de serie de potencias}
Definiciones basadas en: Ecuaciones diferenciales,Dennis G. Zill y Michales R cuellen. cap6, pg220-222\\
 Una serie de potencias de la forma $(x-x_0)$ es una serie infinita de la forma:\\
\begin{align*} 
\Sigma{^\infty_{n=0}}C_n (x-x_0)^n=C_0+C_1(x-x_0)^1+C_2(x-x_0)^2+...
\end{align*}
se dice que esta serie es una \textbf{serie de potencia centrada en $x$} que es con las que se trabajará principalmente en esta sección. A continuación se resume algunos conceptos importantes acerca de las series de potencia: \vspace{0.5cm}\\
\textbf{Convergencia:} una serie de potencias $\Sigma{^\infty_{n=0}}C_n(x-a)^n$ es convergente en un valor especificado de $x$ si su sucesión de sumas parciales ${S_N(x)}$ converge, es decir, si el $\displaystyle\lim_{N \to \infty} S_N(x) = \displaystyle\lim_{N \to \infty} \Sigma^N_{n=0} C_n(x-a)^n$ existe. si el límite no existe en $x$, entonces se dice la que la serie es divergente.\vspace{0.2cm}\\
 \textbf{Intervalo de convergencia:} toda serie tiene uno. El intervalo de convergencia es el conjunto de todos los números reales $x$ para los que converge la serie.\vspace{0.2cm}\\
  \textbf{Radio de convergencia:} toda serie de potencias tiene un radio de convergencia $R$. si $R>0$, entonces la serie de potencias $\Sigma{^\infty_{n=0}}C_n(x-a)^n$ converge para $|x-a|<R$ y diverge para $|x-a|>R$. Si la serie converge sólo en su centro $a$, entonces $R=0$. Si la serie converge para toda $x$, entonces se escribe $R=\infty$. Recuerde que la desigualdad de valor absoluto $|x-a|<R$ es equivalente a la desigualdad simultanea $a-R<x<a+R$. Una serie de potencias podría converger o no en los puntos extremos $a-R$ y $a+R$ de este intervalo.\vspace{0.2cm}\\ 
   \textbf{Convergencia absoluta:} dentro de su intervalo de convergencia, una serie de potencias converge absolutamente. En otras palabras, si $x$ es un número en el intervalo de convergencia y no es un extremo del intervalo, entonces la serie de valores absolutos $\Sigma{^\infty_{n=0}}|C_n(x-a)^n|$ converge.\vspace{0.2cm}\\

  \textbf{Una serie de potencias define una función}: una serie de potencias define una función $f(x)= \Sigma{^\infty_{n=0}} C_n (x - a)_n$ cuyo dominio es el intervalo de convergencia de la serie. Si el radio de convergencia es $R>0$, entonces $f$ es continua, derivable e integrable en el intervalo de $(a - R, a + R)$. Además, $f'(x)$ y $\int f(x)dx$ se encuentran derivando e integrando término a término. La convergencia en un extremo se podría perder por derivación o ganar por integración. Si $y= \Sigma{^\infty_{n=0}} C_n x^n$ es una serie de potencias en $x$, entoces las primeras dos derivadas son $y'=   \Sigma{^\infty_{n=0}} nx ^ { n - 1 }$ y $y''=  \Sigma{^\infty_{n=0}} n ( n - 1 )x^{n-2}$. Observe que el primer término en la primera derivada y los dos primeros términos de la segunda derivada son cero. se omiten estos términos cero y se escribe\\
  \begin{align*}
  y'= \Sigma{^\infty_{n=1}} c_n nx^{n-1} \hspace{0.3cm}  y  \hspace{0.3cm}  y''= \Sigma{^\infty_{n=2}} c_nn(n-1)x^{n-2}
  \end{align*}
 Estos resultados son importantes y se usan en breve.\vspace{0.3cm}\\
 \textbf{Propiedad de identidad} si $\Sigma{^\infty_{n=1}} C_n(x-a)^n = 0, R>0,$ para los números $x$ en el intervalos de convergencia, entonces $C_n = 0$ para toda $n$.\vspace{0.3cm}\\
 \textbf{Analítica en un punto}: una función $f$es analítica en un punto $a$ si se puede representar mediante una serie de potencias en $x - a$ con un radio positivo o infinito de convergencia. En cálculo se ve que las funciones como $e^x, cos x sen x, ln(1 - x)$, entre otras, se pueden reprentar mediante series de Taylor.\\
 \begin{align*}
 e^x = 1 + \frac{x}{1!} + \frac{x^2}{2!} +..., sen x = x- \frac{x^3}{3!} + \frac{x^5}{5!} -..., cos x = 1 - \frac{x^2}{2!} + \frac{x^4}{4!} - \frac{x^6}{6!}...
 \end{align*}
 para $|x|<\infty.$ Estas series de Taylor centradas en 0, llamadas series de Maclaurin, muestran que $e^x, sen x$ y $cos x$ son analíticas en $x = 0$.\vspace{0.3cm}\\
\section{Solución en series de potencias para ecuaciones lineales}
Teoria basada en: Ecuaciones diferenciales,Dennis G. Zill y Michales R cuellen. cap6, pg223-227
\subsection{Suma de dos series de potencias }
Escriba $\Sigma{^\infty_{n=2}} n(n-1)C_nx^{n-2} + \Sigma{^\infty_{n=0}} C_nx^{n+1}$ como una sola serie de potencias cuyo término general implica a $x^k$.\vspace{0.3cm}\\
\textbf{Solución} para sumar las dos series es necesario que ambos índices de las sumas comiencen con el mismo número y las potencias de $x$ en cada caso estén “en fase”; es decir, si una serie comienza con un múltiplo de, por ejemplo, $x$ a la primera potencia, entonces se quiere que la otra serie comience con la misma potencia. Observe que en el problema la primera serie empieza con $x^0$, mientras que la segunda comienza con $x^1$.Si se escribe el primer término de la primera serie fuera de la notación de suma,\\
\begin{align*}
\Sigma{^\infty_{n=2}} n(n-1)C_nx^{n-2} + \Sigma{^\infty_{n=0}} C_nx^{n+1} = 1\cdot 1c_2x^0 + \Sigma{^\infty_{n=3}} n(n-1)C_nx^{n-2} + \Sigma{^\infty_{n=0}} c_nx(n+1),
\end{align*}
vemos que ambas series del lado derecho empiezan con la misma potencia de $x$, en particular $x^1$. Ahora, para obtener el mismo índice de la suma, se toman como guía los exponentes de $x$; se establece $k = n - 2$ en la primera serie y al mismo tiempo $k = n + 1$ en la segunda serie. El lado derecho se convierte en\\
\begin{align*}
2c_2 + \Sigma{^\infty_{k=1}} (k + 2)(k + 1) c_{k+2}x^k + \Sigma{^\infty_{k=1}} c_{k-1}x^k.
\end{align*}
Recuerde que el índice de la suma es una variable “muda”; el hecho de que $k = n - 1$ en un caso y $k = n + 1$ en el otro no debe causar confusión si se considera que lo importante es el \textit{valor} del índice de suma. En ambos casos $k$ toma los mismos valores sucesivos $k = 1, 2, 3$, ... cuando $n$ toma los valores $n = 2, 3, 4,$ ... para $k = n + 1$ y $n = 0, 1, 2,$ ... para $k = n - 1)$. Ahora es posible sumar las series de anteriores término a término:
\begin{align*}
\Sigma{^\infty_{n=2}} n(n-1)c_nx^{n-2} + \Sigma{^\infty_{n=0}} c_nx^{n+1} = 2c_2 + \Sigma{^\infty_{k=1}} [(k+2)(k+1) c_{k+2} + c_{k-1}]x^k.
\end{align*}
Si no está convencido de este resultado, entonces escriba algunos términos de ambos lados de la igualdad.\vspace{0.3cm}\\
\subsection{Puntos ordinarios y singulares}
Suponga que la ecuación diferencial lineal de segundo orden\\
\begin{align}
a_2(x)y'' + a_1(x)y' + a_0(x)y = 0
\end{align}
se escribe en forma estándar\\
\begin{align}
y'' + P(x)y' + Q(x)y = 0
\end{align}
dividiendo entre el coeficiente principal $a_2(x)$. Se tiene la siguiente definición.\\
Se dice que un punto $x_0$ es un \textbf{punto ordinario} de la ecuación diferencial (1) si tanto $P(x)$ como $Q(x)$ en la forma estándar (2) son analíticas en $x_0$. Se dice que un punto que no es punto ordinario es un \textbf{punto singular} de la ecuación.\vspace{0.3cm}\\
Cada valor finito de $x$ es un punto ordinario de la ecuación diferencial $y'' +  (e_x)y'+ (sen x)y = 0$. En particular, $x = 0$ es un punto ordinario porque, como ya se vio en la definición de analítica en un punto, tanto $e_x$ como $sen x$ son analíticas en este punto. Como se meciono en las definiciones iniciales se establece que si por lo menos una de las funciones $P(x) y Q(x)$ en (2) no es analítica en $x_0$, entonces $x_0$ es un punto singular. Observe que $x = 0$ es un punto singular de la ecuación diferencial $y'' + (e_x)y'+ (ln x)y = 0$ porque $Q(x) = ln x$ es discontinua en $x = 0$ y, por tanto, no se puede representar con una serie de potencias en $x$.
\subsection{Teorema 1: existencia de soluciones en series de potencias}
Si $x = x_0$ es un punto ordinario de la ecuación diferencial (1), siempre es posible encontrar dos soluciones linealmente independientes en la forma de una serie de potencias centrada en $x_0$, es decir, $y = \Sigma{^\infty_{n=0}} c_n(x - x_0)^n$. Una solución en serie converge por lo menos en un intervalo definido por $|x - x_0| < R$,donde $R$ es la distancia desde $x_0$ al punto singular más cercano.\vspace{0.3cm}\\
Se dice que una solución de la forma $y = \Sigma{^\infty_{n=0}} c_n(x - x_0)^n$ es una \textbf{solución respecto a un punto ordinario $x_0$}. La distancia $R$ en el teorema anterior es el \textit{valor mínimo
o límite inferior} del radio de convergencia de las soluciones en serie de la ecuación
diferencial respecto a $x_0$.\\
En el ejemplo siguiente, se usa el hecho de que en el plano complejo, la distancia entre dos números complejos $a + bi$ y $c + di$ es exactamente la distancia entre los puntos $(a, b)$ y $(c, d)$.
\subsubsection{Solución con series de potencias. Ejemplo 1}
Resuelva $(x^2 + 1)y'' + xy' - y = 0$\\
\textbf{Solución} como se vio en la definicion de puntos ordinarios y singulares, la ecuación diferencial dad tiene puntos sigulares en $x = \pm  i$ y, por tanto, una solución en serie de potencias centrada en 0 que converge al menos para $|x| < 1$, donde 1 es la distancia en el plano complejo desde 0 a $i$ o $-i$. La suposición $y = \Sigma{^\infty_{n=0}} c_nx^n$ y sus primeras dos derivadas. Teniendo en cuenta serie de la definicion de series de potencia en la página 3. temos que:\vspace{0.3cm}\\
\begin{small}
$(x^2 + 1) \Sigma{^\infty_{n=2}} n (n-1) c_nx^{n-2} + x \Sigma{^\infty_{n=1}} nc_nx^{n-1} - \Sigma{^\infty_{n=0}} c_nx^n\vspace{0.3cm}\\
\hspace*{1cm}= \Sigma{^\infty_{n=2}} n (n-1) c_nx^n + \Sigma{^\infty_{n=2}} n (n-1) c_nx^{n-2} + \Sigma{^\infty_{n=1}} n (n-1) nc_nx^n -\Sigma{^\infty_{n=0}} c_nx^n\vspace{0.3cm}\\
\hspace*{1cm}= 2c_2x^0 - c_0x^0 + 6c_3x + c_1x - c_1x + \Sigma{^\infty_{n=2}} n (n-1) c_nx^n\vspace{0.3cm}\\
\hspace*{1.3cm}+ \Sigma{^\infty_{n=4}} n (n-1) c_nx^{n-2} + \Sigma{^\infty_{n=2}} n c_nx^n - \Sigma{^\infty_{n=2}} c_nx^n\vspace{0.3cm}\\
\hspace*{1cm} = 2c_2 - c_0 + 6 c_3x + \Sigma{^\infty_{k=2}} [k(k - 1)c_k + (k + 2) (k + 1)c_{k+2} + kc_k - c_k] x^k$\vspace{0.3cm}\\
\hspace*{1cm} $ = 2c_2 - c_0 + 6 c_3x + \Sigma{^\infty_{k=2}} [(k - 1)(k - 1)c_k + (k + 2) (k + 1) c_{k+2}]x^k = 0.$\vspace{0.35cm}\\
\end{small}
De esta identidad se concluye que $2c_2 - c_0 =0$, $6c_3 = 0,$ y\vspace{0.3cm}\\
\hspace*{3cm} $(k+1)(k-1) c_k + (k + 2) (k + 1)c_{k+2} = 0.$\vspace{0.3cm}\\
Por tanto \hspace{2.5cm} $c_2 = \frac{1}{2} c_0$\vspace{0.3cm}\\
\hspace*{4.4cm} $c_3 = 0$\vspace{0.3cm}\\
\hspace*{4.4cm} $c_{k+2} = \frac{1 - k}{k + 2}c_k,\hspace{1cm} k=2,3,4...$\vspace{0.3cm}\\
Sustituyendo $k = 2,3,4,...$ en la última fórmula se obtiene\\
\hspace*{3.5cm} $c_4 = - \frac{1}{4} c_2 = - \frac{1}{2*4}c_0 = - \frac{1}{2^22!} c_0$\vspace{0.3cm}\\
\hspace*{3.5cm} $c_5 = - \frac{2}{5} c_3 = 0$\vspace{0.3cm}\\
\hspace*{3.5cm} $c_6 = - \frac{3}{6} c_4 = \frac{3}{2\cdot 4 \cdot 6} c_0 = - \frac{1\cdot3\cdot5}{2^44!} c_0$\vspace{0.3cm}\\
\hspace*{3.5cm} $ c_7 = - \frac{4}{7} c_5 =0$ \vspace{0.3cm}\\
\hspace*{3.5cm} $c_8 = - \frac{5}{8} c_6 = - \frac{3\cdot5}{2\cdot4\cdot6\cdot8\cdot10} c_0 = \frac{1\cdot3\cdot5}{2^44!}c_0$ \vspace{0.3cm}\\
\hspace*{3.5cm} $c_9 = - \frac{6}{9} c_7 = 0$, \vspace{0.3cm}\\
\hspace*{3.5cm} $c_{10} = - \frac{7}{10} c_8 = \frac{3\cdot5\cdot7}{2\cdot4\cdot6\cdot8\cdot10} c_0 = \frac{1\cdot3\cdot5\cdot7}{2^55!} c_0.$ \vspace{0.3cm}\\
Etc. Por tanto,\\
$y = c_0 + c_1x + c_2x^2 + c_3x^3 + c_4x^4 + c_5x^5 + c_6x^6 + c_7x^7 + c_8x^8 + c_9x^9 + c_{10}x^{10} + ...$\vspace{0.3cm}\\ 
 \hspace*{0.3cm} $ =c_0 \left[ 1 + \frac{1}{2} x^2 - \frac{1}{2^22!} x^4 + \frac{1\cdot3}{2^33!}x^6 - \frac{1\cdot3\cdot5}{2^44!}x^8 - \frac{1\cdot3\cdot5\cdot7}{2^55!}x^10 - ... \right] + c_1x$ \vspace{0.3cm}\\
 \hspace*{0.3cm} $= c_0 y_1 (x)$.\vspace{0.3cm}\\
 Las soluciones son el polinomio $y_2 (x) = x$ y la serie de potencias\\
 \begin{align*}
 y_1 (x) = 1 + \frac{1}{2} x^2 + \Sigma{^\infty_{n=2}} (-1)^{n-1} \frac{1\cdot3\cdot5\cdot\cdot\cdot(2n - 3)}{2^nn!} x^{2n}, \hspace{0.7cm} |x| < 1.
 \end{align*}
 Tomado de Ecuaciones diferenciales,Dennis G. Zill y Michales R cuellen. cap6, pg226-227\\
 \textbf{Pasos para solucionar el ejercicio}\vspace{0.3cm}\\
 \hspace*{0.3cm} \textbf{1} Partimos de que $y = \Sigma{^\infty_{n=0}} C_n X^n$ y le realizamos las derivadas que hayan en la ecuación.\vspace{0.3cm}\\
 \hspace*{0.3cm} \textbf{2} Sustituimos las sumatorias y sus derivadas donde sea necesario; reordenamos terminos.\vspace{0.3cm}\\
  \hspace*{0.3cm} \textbf{3} Operamos los terinos que sea posible. Por ejemplo, se suman los exponentes. De manera que todo quede en terminos de sumas.\vspace{0.3cm}\\
  \hspace*{0.3cm} \textbf{4} Una vez operamos, podemos ver que no totdas las sumas empiezan en 0, ni todas tienen las mismas potencias. Entonces buscamos que los valores de $n$ para la sumatoria sean el mismo y que las potencias de $x$ sean igual (por lo general se igualan al mayor, en este caso, n) y si es de nuestras prefencia se trabaja con otra variable para no crear confuciones, en este caso se reemplaza por $k$.\vspace{0.3cm}\\
  \hspace*{0.3cm} \textbf{5} Ahora se busca que todos los coeficientes queden en terminos uno de ellos, en este caso buscamos que queden en terminos de $c_0$ que es nuestro coeficiente conocido.(no hay una formula para hacerlo, simplemete se hace de acuerdo a lo que tengamos)\vspace{0.3cm}\\
  \hspace*{0.3cm} \textbf{6} Se reescriben los terminos como estaban inicalmente. Por ejemplo, $y= C_0 + C_1x+ C_2x^2 + C_3x^3$\vspace{0.3cm}\\
  \hspace*{0.3cm} \textbf{7} Sustituimos los valores de los coeficientes (excepto $c_o$).\vspace{0.3cm}\\
   \hspace*{0.3cm} \textbf{8} El valor al cual se igualó los coeficientes en el paso 5 es un valor comun en los demas terminos. Por lo tanto se simplifica la expresion factorizando.\vspace{0.3cm}\\
   \hspace*{0.3cm} \textbf{9} La expresion que se obtenga es la solución.\vspace{0.3cm}\\
   
   
\section{Programación en Matlab}
En esta parte del artículo se expresa el codigo de los ejercicios programados en Matlab, cabe resaltar que son diferentes al ejemplo 1, no obstante tambien cumplen con las especificaciones del tema explicado.\\
Por otra parte cabe aclarar que el primer ejercicio de programacíon fue tomados del libro de Ecuaciones diferenciales,Dennis G. Zill y Michales R cuellen. Ejercicio 33,pg230. Además se soluciona un ejercicio de aplicación que consiste en encontrar la posiscion de un péndulo en un tiempo dado. Para cada uno de los ejericios se especifican las condiciones mediante codigos, por ejemplo, el intervalo de tiempo etc. Por último la programación de ambos ejercicios se realizó bajo el lenguaje ODE45 propio del software Matlab.\\
\end{document}













